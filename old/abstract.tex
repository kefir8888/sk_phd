Modern agriculture is characterized by scale and efficiency.
Autonomous disease detection as well as autonomous yield estimation are crucial to the automatization, loss prevention and further efficiency improvement.
As some of these labour-intensive tasks are being automated, certain technical problems should be solved.
In this work a complex solution is presented, including a robot with omnidirectional shassis and a camera setup, and a set of algorithms that solve the problems of powdery mildew identification and yield volume estimation for ellipsoidal fruit and vegetables.
An improvement was proposed to the algorithm of ellipsoidal model fitting to the point cloud data. Specifically, the process of checking the hypothesis was sped up.
All the proposed algorithms were extensively tested on real-world data and evaluated in terms of the metrics relevant to the industry.
The results suggest the applicability of the proposed approaches in the real-world scenarious.

% In this work a random sample consensus-based method of second order curve recognition is examined in terms of the dependence of the output quality on the intensity of noise in the input data. Specifically, the recognition method is applied to ellipses. The quality is evaluated the intersection over union between the real ellipse and the predicted one. The results quantitively describe the dependence of the output quality on the characteristics of input data, suggesting deterioration of the model as the data becomes more and more noisy. A series of numerical experiments is conducted on challenging synthetic data.

