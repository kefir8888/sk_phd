% -*-latex-*-
% NOTE:
% These templates make an effort to conform to the Skoltech Thesis specifications,
% however the specifications can change.  We recommend that you verify the
% layout of your title page with your thesis advisor and the Education department 
% before printing your final copy.
\title{Thesis Template Skoltech}

\author{Student Name}
% If you wish to list your previous degrees on the cover page, use the 
% previous degrees command:
%       \prevdegrees{MSc, University of Salzburg (2007)}
% You can use the \\ command to list multiple previous degrees
%       \prevdegrees{B.S., University of California (1978) \\
%                    S.M., Massachusetts Institute of Technology (1981)}
\department{Skoltech Center of Research, Entrepreneurship and Innovation}

\degree{Doctoral Program in Engineering Systems}

\degreemonth{November}
\degreeyear{2019}
\thesisdate{November 2019}

%% By default, the thesis will be copyrighted to MIT.  If you need to copyright
%% the thesis to yourself, just specify the `vi' documentclass option.  If for
%% some reason you want to exactly specify the copyright notice text, you can
%% use the \copyrightnoticetext command.  
%\copyrightnoticetext{\copyright \@author \@degreeyear}

% If there is more than one supervisor, use the \supervisor command
% once for each.
\supervisor{Edward Crawley}{Founding President, Professor}

% This is the department committee chairman, not the thesis committee
% chairman.  You should replace this with your Department's Committee
% Chairman.
%\chairman{Name}{Title}

% Make the titlepage based on the above information.  If you need
% something special and can't use the standard form, you can specify
% the exact text of the titlepage yourself.  Put it in a titlepage
% environment and leave blank lines where you want vertical space.
% The spaces will be adjusted to fill the entire page.  The dotted
% lines for the signatures are made with the \signature command.
\maketitle

% The abstractpage environment sets up everything on the page except
% the text itself.  The title and other header material are put at the
% top of the page, and the supervisors are listed at the bottom.  A
% new page is begun both before and after.  Of course, an abstract may
% be more than one page itself.  If you need more control over the
% format of the page, you can use the abstract environment, which puts
% the word "Abstract" at the beginning and single spaces its text.

%% You can either \input (*not* \include) your abstract file, or you can put
%% the text of the abstract directly between the \begin{abstractpage} and
%% \end{abstractpage} commands.

% First copy: start a new page, and save the page number.
\cleardoublepage
% Uncomment the next line if you do NOT want a page number on your
% abstract and acknowledgments pages.
% \pagestyle{empty}
\setcounter{savepage}{\thepage}
\begin{abstractpage}
In short this is the content of my work...
\end{abstractpage}


\clearpage
\section*{Publications}
\subsection*{Main author}
\nobibliography*

\begin{enumerate}
    \item \bibentry{Skoltech2017}
\end{enumerate}

\subsection*{Co-author}

\begin{enumerate}
    \item \bibentry{Skoltech2017}
\end{enumerate}

\cleardoublepage

\begin{dedication}
Dedicated to my parents.
\end{dedication}

\section*{Acknowledgments}

Let me thank to all my supporters ...
