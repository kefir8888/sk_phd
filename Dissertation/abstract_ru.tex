\chapter*{Аннотация}

Современному сельскому хозяйству свойственны промышленные масштабы и высокая эффективность.
Автономное обнаружение заболеваний и оценка урожая критически важны для автоматизации, предотвращения потерь продукции и дальнейшего повышения эффективности производства.
В процессе автоматизации этих процессов встают некоторые технические задачи, которым и посвящена эта диссертация.
В этой работе представлено комплексное решение, включающее в себя робота на всенаправленном колесном шасси, систему камер для мониторинга растений в условиях современного тепличного хозяйства, а также набор алгоритмов, решающих задачи детектирования мучнистой росы и оценки объема эллипсоидальных фруктов и овощей.
Была предложена модификация алгоритма построения эллипсоида по зашумленному облаку точек, а именно процесс оценки гипотез был сведен к матричному умножению с целью ускорения.
Все предложенные алгоритмы были широко протестированы на реальных данных и были оценены по метрикам, релевантным для сельского хозяйства.
Результаты показывают, что предложенные методы могут быть внедрены в практику в современных тепличных хозяйствах.
