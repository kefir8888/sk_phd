\chapter*{Introduction}                         % Заголовок
\addcontentsline{toc}{chapter}{Introduction}    % Добавляем его в оглавление

\newcommand{\actuality}{}
\newcommand{\progress}{}
\newcommand{\aim}{{\textbf\aimTXT}}
\newcommand{\tasks}{\textbf{\tasksTXT}}
\newcommand{\novelty}{\textbf{\noveltyTXT}}
\newcommand{\influence}{\textbf{\influenceTXT}}
\newcommand{\methods}{\textbf{\methodsTXT}}
\newcommand{\defpositions}{\textbf{\defpositionsTXT}}
\newcommand{\reliability}{\textbf{\reliabilityTXT}}
\newcommand{\probation}{\textbf{\probationTXT}}
\newcommand{\contribution}{\textbf{\contributionTXT}}
\newcommand{\publications}{\textbf{\publicationsTXT}}

%\input{common/actuality}
%\input{common/characteristic}

\textbf{The volume and structure of the work.} 
The dissertation consists of~an abstract, introduction,
\formbytotal{totalchapter}{chapter}{}{s}{},
conclusions, lists of abbreviations, figures, tables and \formbytotal{totalappendix}{appendix}{}{es}{}.
The full thesis' volume is
\formbytotal{TotPages}{page}{}{s}{s}, including
\formbytotal{totalcount@figure}{figure}{}{s}{} and
\formbytotal{totalcount@table}{table}{}{s}{}.
%TODO: из-за костылей с разбиением на ru и eng список литературы, эта штука может неправильно обрабатывать списки. Внимательно проверяйте ее и, при необходимости, проописывайте руками нужное число.
%The list of references consists of \formbytotal{citenum}{source}{}{s}{}.
The list of references consists of 66 sources.

\section{Relevance}

Modern farming is characterized by both scale and efficiency.
Both precise yield estimation and disease detection are its integral parts.
Tangerines and tomatoes are one of the most important fruit in the agricultural produce worldwide \cite{li2025metafruit}.

More specifically, knowing the mass of the tomatoes that are ready to be picked is crucial for proper logistics planning \cite{tahir2021comprehensive}.
And volume is one of the most important physical attributes of the fruit yield in agricultural production.
If performed manually, it is time-consuming, expensive and prone to error due to the loss of focus under the monotonous conditions of such labour.
In order to automate this process, computer vision-based systems are often used.
Because of the scale of modern farms and greenhouses, covering them entirely with stationary cameras is infeasible.
Thus, the majority of the solutions are based on autonomous robots, i.e. flying or wheeled.

This work addresses the problem of estimating position, size and orientation of tomatoes and tangerines under the conditions close to the real greenhouse.
This problem is addressed with two different approaches, one relying on classical algorythmical surface fitting, and another one being an end-to-end Neural Network-based approach.

Secondly, disease detection is crucial in loss prevention and taking timely measures to optimize the production.
There are multiple diseases that manifest in a visible way, with powdery mildew being one of them.
In this work the the problem of powdery mildew identification is addressed.
In order to solve it, a robot was developed, as well as a multi-camera setup for data collection.

\section{Dissertation Goals}

The goal of this dissertation is to investigate the performance of volume estimation algorithms that process 3D point cloud data in application to ellipsoidal objects in agricultural setting.
The second goal it to evaluate the performance of the disease identification algorithms in application to the tomato plants.

In order to achieve these goals, the following problems are solved.

Thie first is to develop an implementation of the ellipsoid fitting algorithm with point cloud input.
The second is to evaluate its performance in multiple settings, including various data collection conditions and different objects.
The third is to solve the problem of volume estimation with an end-to-end approach, i.e. Neural Network.
The fourth is to perform data collection in real greenhouse conditions, as well as markup, data coherency evaluation and model training for the problem of disease detection.
The fifth is to develop a robot capable of moving in the greenhouse on two types of surface, while carrying a computer and a camera setup, with sufficiently long battery lifetime.

\section{Research methodology}

The methodology incorporates theoretical ideas from the field of pattern recognition.
The design of the main algorithms is based on random sample consensus-based approaches, as well as the properties of the second-order curves and surfaces.
At the same time, a lot of the specific design solutions both in terms of the software and the hardware were made in accordance with the real-world demands from the engineering standpoint.
It provided the means to consider the prospective usage in terms of vibration reliability, autonomous operation time, and locomotion abilities of the robot.
The following is a list of main software tools and frameworks used to conduct numerical experiments:
\begin{itemize}
    \item OpenCV
    \item Open3D
    \item ROS2
    \item Matlab
    \item PyTorch
\end{itemize}

\section{Validity of the obtained results}

The obtained results were verified by an array of numerical experiments on both real-world and synthetic data, see the respective sections in the relevant chapters.

\section{Approbation}

The results were presented on 4 conferences: MIPT conference, Zavalishinsky readings, and two papers at EDM conference.

\section{Personal contribution of the author}

The author actively contributed to several research activities.

First, the autor formulated the research hypothesis, proposed and validated experimental setups, as well as participated in experiments, including ones in the real greenhouse environment.

Second, the author participated in the algorithms development and numerical experiments.

Third, the author took part in the designing, manufacturing, and overall working on the omnidirectional robot, including milling, assembly, painting, components purchase, cable management, soldering.

Fourth, the author participated in writing all 6 papers, that are the foundation of the thesis presented.

Fifth, the author made significant contribution to the legal part of the project, in particular grant-related activities, reports, patenting, and TRL.

Sixth, the author contributed to the organization of the project, including personnel search, trips, and procurement.

\section{Propositions for Defense}

\begin{enumerate}
    \item The problem of model evaluation for the second order surface fitting was reduced to a matrix multiplication. The experimental results suggest that the performance of the proposed method on a single modern laptop allows for the real-time greenhouse monitoring.
    \item The influence of the noise on the quality of ellipsoid fitting was numerically evaluated. The practical outcomes include results for the second-order curves as a simpler object.
    \item A unique dataset of tomato leaves infected by powdery mildew was collected, marked and patented. The defining feature is that the images were taken with a user-grade camera from a moving robot in the real greenhouse. The publicly available datasets are all filmed in lab conditions with artificial lighting, steady objects and expensive cameras.
    \item A robot was developed and its crucial feature (the wheel) was patented. The robot was specifically designed to serve as a hardware support for the whole project, and it encompasses the camera setup, the computing capabilities onboard, and a prolongued lifetime on a single charge.
\end{enumerate}

\section{Engineering Novelty and Practical Outcomes}

In terms of the practical results, the following main results should be highlighted.

First, a problem of tomato and tangerine volume estimation was solved on the real point cloud data.
Second, a problem of powdery mildew identification on tomato leaves was solved on real RGB data from user-grade cameras.
Third, PointNet was adapted to the regression problem and used to estimate the volume of tangerines in the end-to-end fashion

The robot that was developed during the research was certified to be TRL-5 (Technology Readiness Level) in terms of its ability to move on both flat surface and rails.
The robot is also capable of using its kinematic model to follow any curve on a two-dimensional surface.
Practically it means that the platform is ready to carry scientific or industrial load, if controlled in the environment properly.

Three patents were issued to Skoltech during the research.

The first is for a dataset of tomato leaves infected by powdery mildew. The unique feature is that the images were filmed with a mid-price RGB camera from a moving robot in the real greenhouse, which is in great contrast with the publicly available datasets, filmed in the lab, with artificial lighting, steady scene and expensive camera setups.

The second is for a computer program capable of classifying images into infected by powdery mildew and not infected.

The third is for a wheel, corresponding of a Mecanum wheel, rail wheel, mounted coaxially, integrated motor, and a suspension system.

Finally, a company AIDA Robotics is in process of being established.